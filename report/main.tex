\documentclass[12pt]{article}
\usepackage[utf8]{inputenc}
\usepackage[french]{babel}
\usepackage{graphicx}
\usepackage{fancyhdr}
\usepackage[table]{xcolor}
\usepackage[T1]{fontenc}
\usepackage{float}
\usepackage[noend]{algpseudocode}
\usepackage{algorithm}
\usepackage[colorinlistoftodos]{todonotes}
\usepackage[colorlinks=true, allcolors=blue]{hyperref}
%\usepackage{amsmath}

\title{Rapport de Projet}					
\author{hanouticelina}								 
\date{Décembre 2019}

\makeatletter
\let\thetitle\@title
\let\theauthor\@author
\let\thedate\@date
\makeatother

\begin{document}
\begin{titlepage}
	\centering
    \vspace{0.7cm}
    \includegraphics[scale = 0.4]{sorbonne.png}\\[2cm]
    \textbf{\LARGE MOGPL}\\[0.4 cm]
    \Large{Modélisations, Optimisation, Graphes\\et Programmation Linéaire}\\[1 cm]
	\rule{\linewidth}{0.2 mm} \\[0.4 cm]
	{\Huge{\bfseries \thetitle}}\\
	\rule{\linewidth}{0.2 mm} \\[1.5 cm]
	
	\begin{minipage}{0.4\textwidth}
		\begin{flushleft} \large
			\emph{Étudiants:}\\
			Celina HANOUTI\\
			Idles MAMOU
			\end{flushleft}
			\end{minipage}~
			\begin{minipage}{0.4\textwidth}
			\begin{flushright} \large
			\emph{Numéros Étudiants:} \\
			3522046\\
			à compléter
		\end{flushright}
	\end{minipage}\\[4 cm]
	{\large \thedate}\\[3 cm]
	\vfill
\end{titlepage}
\tableofcontents
\newpage

\section{Introduction}
On considère un jeu de dés à deux joueurs qui consiste, à chaque tour, à lancer entre 1 et $D$ dés à six faces, le but du jeu étant d'atteindre en premier $N$ points.
\begin{itemize}
  \item Si au moins l'un des dés tombe sur 1, le nombre de points marqués lors du tour est 1.
  \item Sinon le nombre de points est la somme des dés lancés.
\end{itemize}
Le but de ce projet est de trouver une stratégie optimale à ce jeu en considérant l'aspect théorique et pratique du problème et de comparer cette dernière à d'autre stratégie de l'état de l'art.
Par ailleurs on étudiera deux variantes du jeu: la variante séquentielle et la variante simultanée.

\newpage
\section{Probabilités}
Lorsqu'un joueur lance $d$ $\geq$ 1 dés, soit il obtient un seul point, soit il obtient entre 2$d$ et 6$d$ (au pire, les $d$ dés tomberont tous sur la face 2 et au mieux, les $d$ dés tomberont tous sur la face 6).
\newline
On définit $P(d,k)$ comme étant la probabilité qu'un joueur qui lance $d$ dés obtienne $k$ points.
La probabilité d'obtenir un point ($k$ = 1) est la probabilité qu'aucun dés lancés ne tombe sur la face 1, on a donc : $P(d,1)$ = 1 - $\left(\frac{5}{6}\right)^d$.
\newline
Comme les dés sont numérotés de 1 à 6, il est évident que le maximum de points que l'on peut obtenir en lançant $d$ dés est 6$d$, il est cependant impossible d'obtenir un score inférieur à 2$d$ ceci s'explique par le fait qu'il n'existe pas de vecteur [$x_1$,$x_2$, .. ,$x_d$] strictement positif tel que:
\newline
$\forall i, 1 \leq i \leq d, x_i \neq 1$ et $x_1+x_2+ .. +x_d = k$ pour $k < 2d$. 
\newline
On a donc : $P(d,k) = 0$ pour $2 \leq k \leq 2d-1$ et $k > 6$.
Pour $2d \leq k \leq 6d$ et $d$ $geq$ 2, on définit $Q(d,k)$ comme étant la probabilité conditionnelle d'obtenir $k$ points en jetant $d$ dés sachant qu'aucun dé n'est tombé sur 1.
On peut définir cette probabilité récursivement, en effet, une interprétation ensembliste de la formule est peut être plus parlante que la formule elle-même:
\newline
considérons les évènements $E_k^d$ = "obtenir $k$ points en lançant $d$ dés sachant qu'aucun dé n'est tombé sur 1".
\newline
et $F_k$ = "obtenir $k$ points au $d$-ième lancé".
\\
\newline
On a donc $E_k^d$ = $\bigcup_{j=2}^{6}$ ($E_{k-j}^{d-1} \cap F_j$).
\newline
\\
De plus,
$P(F_j) = \frac{1}{5}$ pour tout $2 \leq j \leq 6$ et $P(E_k^d) = Q(d,k)$.
\newline
\\
On a bien la formule $Q(d,k) = \sum_{j=2}^6 \frac{Q(d-1,k-j)}{5}$.
\\
et on a : $P(d,k) = \left(\frac{5}{6}\right)^d Q(d,k)$.
\newline
\\
Les cas d'initialisation sont les cas où $d$ = 1 auquel cas, $Q(d,k)$ = $\frac{1}{5}$, $k$ < 2$d$ ou $k$ > 6$d$ auquel cas $Q(d,k)$ = 0.

\begin{algorithm}
\caption{}
\begin{algorithmic}[1]
\Require 
\Function{}{}
\EndFunction
\Statex

\end{algorithmic}
\end{algorithm}
\newline
\\
\section{Variante séquentielle}
\subsection{Stratégie aveugle}
À chaque tour, chacun des joueurs devra choisir le nombre de dés à lancer, en effet, plus on lance de dés, plus on augmente la probabilité que l'un d'eux tombe sur 1, et moins on en lance, moins on a de chance d'obtenir un large score. On considère ici une stratégie simple, qui consiste à choisir un nombre de dés qui maximise en moyenne le nombre de points. Pour un lancé, si celui-ci ne tombe pas sur la face 1, on gagne en moyenne 4 points, sur $d$ lancés, on gagne donc en moyenne 4$d$ points avec une probabilité de $\left(\frac{5}{6}\right)^d$ et on gagne 1 point avec une probabilité de 1 - $\left(\frac{5}{6}\right)^d$. L'ésperance $EP(d)$ du nombre de points obtenus s'obtient avec la formule suivante: 
\newline
\\
$$EP(d) = 4d\left(\frac{5}{6}\right)^d + 1 - \left(\frac{5}{6}\right)^d$$
\newline
\\
Q2) L'expression est maximale pour $d$ = 5 indépendemment de $D$ (il suffit d'annuler la dérivée par rapport à $d$, on trouve approximativement 5).
\newline
\\
Q3) Si on se place dans le cas où $D$ = 10, on a $d^*(10) = 5$, si lors des 4 premiers lancés, on tombe sur 4 fois la face 6, on sera tout de même obligé de lancer un 5ème lancé et risquer de perdre les 24 points (La probabilité de tomber sur la face 1 n'étant pas nulle). (Il existe plusieurs autres exemples où le joueur n'a pas intérêt à utiliser cette stratégie).
\subsection{Programmation dynamique}
Si $i$ $\geq$ 100 et j < 100 : $EG(i,j) = 1$ et $EG(j,i) = -1$ 
\\
de façon symétrique,
\\
Si $j$ $\geq$ 100 et i < 100 : $EG(j,i) = 1$ et $EG(i,j) = -1$ 
\\
(Principe d'un jeu à somme nulle).
\\
Sinon:
\\
$EG(i,j) = G(i,j) - G(j,i)$
\\
avec $G(i,j)$ la probabilité que le joueur 1 gagne à l'état $(i,j)$.
\\
Si $i$ $\geq$ 100 et j < 100 : $G(i,j) = 1$ et $G(j,i) = 0$.
$$G(i,j) = \max_d \left( P(d,1)(1-G(j,i-1)) + \sum_{k=2d}^{6d} P(d,k)(1-G(j,i-k))\right)$$
$1-G(j,i-k)$ est la probabilité que le joueur adverse ne gagne pas à l'état $(j,i-k)$. On part du principe qu'on arrive à l'état $(i,j)$ en prenant en compte toutes les issues possibles (par rapport au score) en lançant $d$ dés. Je ne suis pas très sure de cette formule, et je n'ai pas d'idée pour l'instant pour la recurrence de EG.


\section{Variante simultanée}
\subsection{Jeu en un coup}
On définit la stratégie de chaque joueur $i$ avec un vecteur $[p_i(1),.. p_i(D)]$ où $p_i(d)$ est la probabilité avec laquelle nous lançons $d$ dés. $p_i(d) \geq 0 $ et $\sum_{d=1}^D p_i(d) = 1$ pour chaque $i \in {1,2}$.

\newline
On considère une approche simplifiée où l'on lance qu'une seule fois les dés.
\newline
Q10) $EG_1(d_1,d_2)$ est l'espérance de gain du joueur 1 s'il lance $d_1$ dés alors que le joueur 2 en lance $d_2$, soit $k$ $\in [1,..,6d_1]$ et $l$ $\in [1,..,6d_2]$ obtenus par le joueur 1 et le joueur 2 réspectivement. Le gain du joueur 1 est +1 si $k > l$, 0 si $k = l$ et -1 si $k < l$. L'espérance de gain est donc donné par la formule suivante:
$$EG_1(d_1,d_2) = \sum_{k=1}^{6d_1} \sum_{l=1}^{k-1} P(d_1,k)P(d_2,l) - \sum_{k=1}^{6d_1} \sum_{l=k+1}^{6d_2} P(d_1,k)P(d_2,l)$$
\newline

Q11) On sait que l'espérance de gain du joueur 1 s'écrit de la manière suivante:
$$p_1^T EG_1 p_2 = \sum_{i=1}^D \sum_{j=1}^D p_1(i)EG_1(i,j)p_2(j)$$
Si le joueur 2 connait la stratégie du joueur 1(ie le vecteur $[p_1(1),.. p_2(D)]$), donc sa stratégie optimale consiste à minimiser l'espérance de gain du joueur 1, c'est à dire:
$$ \min_{p_2} \sum_{i=1}^D \sum_{j=1}^D p_1(i)EG_1(i,j)p_2(j)$$

\newline
Q12) Du point de vue du joueur 1, la stratégie du joueur 2 est inconnue mais nous la supposons optimale ainsi la stratégie optimale du joueur 1 consiste à : 
$$max_{p_1} min_{p_2} p_1^T EG_1 p_2$$
On sait également qu'en optimisant sa stratégie, ce dernier peut omettre les stratégies mixtes du joueur 2 donc la stratégie optimale du joueur 1 peut s'écrire : 
$$max_{p_1} min_{j} p_1^T EG_1^j = max_{p_1} min_{j} \{\sum_{i=1}^D p_1(i)EG_1(i,j)\}$$ 
avec $EG_1^j$ la colonne $j$ de $EG_1$
Le programme linéaire à résoudre pour le joueur 1 est:
$$ max \mbox{ z} $$
s.t
$$ z \leq \sum_{i=1}^D p_1(i) EG_1(i,j) \mbox{ pour tout } j \in \{1,..,D\}$$
$$\sum_{i=1}^D p_1(i) = 1$$


\subsection{Cas général}

Q15)$$
EG_1(i,j) = \left\{
    \begin{array}{ll}
        +1 & \mbox{si } i \geq N \\
        -1 & \mbox{si } j \geq N
    \end{array}
\right.
$$

Q16)
$$E_1^{d_1,d_2}(i,j) = \sum_{k=1}^{6d_1} \sum_{l=1}^{6d_2} EG(k+i,l+j)P(d_1,k)P(d_2,l)$$

Q18) 
$$EG_1(i,j) = \sum_{d_1=1}^{D} \sum_{d_2=1}^{D}p_1(d_1)p_2(d_2)E_1^{d_1,d_2}(i,j)$$
\end{document}
